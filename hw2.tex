%% HOW TO USE THIS TEMPLATE:
%%
%% Ensure that you replace "YOUR NAME" with your own name, in the
%% \studentname command below.  Also ensure that the "answers" option
%% appears within the square brackets of the \documentclass command,
%% otherwise latex will suppress your solutions when compiling.
%% 
%% Type your solution to each problem part within
%% the \begin{solution} \end{solution} environment immediately
%% following it.  Use any of the macros or notation from the
%% header.tex that you need, or use your own (but try to stay
%% consistent with the notation used in the problem).
%%
%% If you have problems compiling this file, you may lack the
%% header.tex file (available on the course web page), or your system
%% may lack some LaTeX packages.  The "exam" package (required) is
%% available at:
%%
%% http://mirror.ctan.org/macros/latex/contrib/exam/exam.cls
%%
%% Other packages can be found at ctan.org, or you may just comment
%% them out (only the exam and ams* packages are absolutely required).


% the "answers" option causes the solutions to be printed
%\documentclass[11pt,addpoints]{exam}
\documentclass[11pt,addpoints,answers]{exam}

%\input{hw2sol}

% required macros -- get header.tex file from course web page

\usepackage{amsmath,amsfonts,amssymb,amsthm}
\usepackage{fullpage}
\usepackage{times}
\usepackage{hyperref}
\usepackage{pdfsync}
\usepackage{microtype}
\usepackage{color}
\usepackage{cleveref}
\crefformat{footnote}{#2\footnotemark[#1]#3}
\definecolor{light-gray}{gray}{0.5}

%%% BLACKBOARD SYMBOLS

\newcommand{\C}{\ensuremath{\mathbb{C}}}
\newcommand{\D}{\ensuremath{\mathbb{D}}}
\newcommand{\F}{\ensuremath{\mathbb{F}}}
\newcommand{\G}{\ensuremath{\mathbb{G}}}
\newcommand{\J}{\ensuremath{\mathbb{J}}}
\newcommand{\N}{\ensuremath{\mathbb{N}}}
\newcommand{\Q}{\ensuremath{\mathbb{Q}}}
\newcommand{\R}{\ensuremath{\mathbb{R}}}
\newcommand{\T}{\ensuremath{\mathbb{T}}}
\newcommand{\Z}{\ensuremath{\mathbb{Z}}}
\newcommand{\QR}{\ensuremath{\mathbb{QR}}}

\newcommand{\Zt}{\ensuremath{\Z_t}}
\newcommand{\Zp}{\ensuremath{\Z_p}}
\newcommand{\Zq}{\ensuremath{\Z_q}}
\newcommand{\ZN}{\ensuremath{\Z_N}}
\newcommand{\Zps}{\ensuremath{\Z_p^*}}
\newcommand{\ZNs}{\ensuremath{\Z_N^*}}
\newcommand{\JN}{\ensuremath{\J_N}}
\newcommand{\QRN}{\ensuremath{\QR_{N}}}
\newcommand{\QRp}{\ensuremath{\QR_{p}}}

%%% THEOREM COMMANDS

\theoremstyle{plain}            % following are "theorem" style
\newtheorem{theorem}{Theorem}[section]
\newtheorem{lemma}[theorem]{Lemma}
\newtheorem{corollary}[theorem]{Corollary}
\newtheorem{proposition}[theorem]{Proposition}
\newtheorem{claim}[theorem]{Claim}
\newtheorem{fact}[theorem]{Fact}

\theoremstyle{definition}       % following are def style
\newtheorem{definition}[theorem]{Definition}
\newtheorem{conjecture}[theorem]{Conjecture}
\newtheorem{example}[theorem]{Example}
\newtheorem{protocol}[theorem]{Protocol}

\theoremstyle{remark}           % following are remark style
\newtheorem{remark}[theorem]{Remark}
\newtheorem{note}[theorem]{Note}
\newtheorem{exercise}[theorem]{Exercise}

% equation numbering style
\numberwithin{equation}{section}

%%% GENERAL COMPUTING

\newcommand{\bit}{\ensuremath{\set{0,1}}}
\newcommand{\pmone}{\ensuremath{\set{-1,1}}}

% asymptotics
\DeclareMathOperator{\poly}{poly}
\DeclareMathOperator{\polylog}{polylog}
\DeclareMathOperator{\negl}{negl}
\newcommand{\Otil}{\ensuremath{\tilde{O}}}

% probability/distribution stuff
\DeclareMathOperator*{\E}{E}
\DeclareMathOperator*{\Var}{Var}

\newcommand{\eps}{\varepsilon}


% sets in calligraphic type
\newcommand{\calA}{\ensuremath{\mathcal{A}}}
\newcommand{\calD}{\ensuremath{\mathcal{D}}}
\newcommand{\calF}{\ensuremath{\mathcal{F}}}
\newcommand{\calH}{\ensuremath{\mathcal{H}}}
\newcommand{\calK}{\ensuremath{\mathcal{K}}}
\newcommand{\calM}{\ensuremath{\mathcal{M}}}
\newcommand{\calX}{\ensuremath{\mathcal{X}}}
\newcommand{\calY}{\ensuremath{\mathcal{Y}}}

% types of indistinguishability
\newcommand{\compind}{\ensuremath{\stackrel{c}{\approx}}}
\newcommand{\statind}{\ensuremath{\stackrel{s}{\approx}}}
\newcommand{\perfind}{\ensuremath{\equiv}}

% font for general-purpose algorithms
\newcommand{\algo}[1]{\ensuremath{\mathsf{#1}}}
% font for general-purpose computational problems
\newcommand{\problem}[1]{\ensuremath{\mathsf{#1}}}
% font for complexity classes
\newcommand{\class}[1]{\ensuremath{\mathsf{#1}}}

% complexity classes and languages
\renewcommand{\P}{\class{P}}
\newcommand{\BPP}{\class{BPP}}
\newcommand{\NP}{\class{NP}}
\newcommand{\coNP}{\class{coNP}}
\newcommand{\AM}{\class{AM}}
\newcommand{\coAM}{\class{coAM}}
\newcommand{\IP}{\class{IP}}

%%% "LEFT-RIGHT" PAIRS OF SYMBOLS

% inner product
\newcommand{\inner}[1]{\langle{#1}\rangle}
\newcommand{\innerfit}[1]{\left\langle{#1}\right\rangle}
% absolute value
\newcommand{\abs}[1]{\lvert{#1}\rvert}
\newcommand{\absfit}[1]{\left\lvert{#1}\right\rvert}
% a set
\newcommand{\set}[1]{\{{#1}\}}
\newcommand{\setfit}[1]{\left\{{#1}\right\}}
% parens
\newcommand{\parens}[1]{({#1})}
\newcommand{\parensfit}[1]{\left({#1}\right)}
% tuple = alias for parens
\newcommand{\tuple}[1]{\parens{#1}}
\newcommand{\tuplefit}[1]{\parensfit{#1}}
% square brackets
\newcommand{\bracks}[1]{[{#1}]}
\newcommand{\bracksfit}[1]{\left[{#1}\right]}
% rounding off
\newcommand{\round}[1]{\lfloor{#1}\rceil}
% floor function
\newcommand{\floor}[1]{\lfloor{#1}\rfloor}
% ceiling function
\newcommand{\ceil}[1]{\lceil{#1}\rceil}
% length of a string
\newcommand{\len}[1]{\lvert{#1}\rvert}
\newcommand{\lenfit}[1]{\left\lvert{#1}\right\rvert}
% length of some vector, element
\newcommand{\length}[1]{\lVert{#1}\rVert}
\newcommand{\lengthfit}[1]{\left\lVert{#1}\right\rVert}

%%% CRYPTO-RELATED NOTATION

% KEYS AND RELATED

\newcommand{\key}[1]{\ensuremath{#1}}

\newcommand{\pk}{\key{pk}}
\newcommand{\vk}{\key{vk}}
\newcommand{\sk}{\key{sk}}
\newcommand{\mpk}{\key{mpk}}
\newcommand{\msk}{\key{msk}}
\newcommand{\fk}{\key{fk}}
\newcommand{\id}{id}
\newcommand{\keyspace}{\ensuremath{\mathcal{K}}}
\newcommand{\msgspace}{\ensuremath{\mathcal{M}}}
\newcommand{\ctspace}{\ensuremath{\mathcal{C}}}
\newcommand{\tagspace}{\ensuremath{\mathcal{T}}}
\newcommand{\idspace}{\ensuremath{\mathcal{ID}}}

\newcommand{\concat}{\ensuremath{\|}}

% GAMES

% advantage
\newcommand{\advan}{\ensuremath{\mathbf{Adv}}}

% different attack models
\newcommand{\attack}[1]{\ensuremath{\text{#1}}}

\newcommand{\atk}{\attack{atk}} % dummy attack
\newcommand{\indcpa}{\attack{ind-cpa}}
\newcommand{\indcca}{\attack{ind-cca}}
\newcommand{\anocpa}{\attack{ano-cpa}} % anonymous
\newcommand{\anocca}{\attack{ano-cca}}
\newcommand{\euacma}{\attack{eu-acma}} % forgery: adaptive chosen-message
\newcommand{\euscma}{\attack{eu-scma}} % forgery: static chosen-message
\newcommand{\suacma}{\attack{su-acma}} % strongly unforgeable

% ADVERSARIES
\newcommand{\attacker}[1]{\ensuremath{\mathcal{#1}}}

\newcommand{\Adv}{\attacker{A}}
\newcommand{\AdvA}{\attacker{A}}
\newcommand{\AdvB}{\attacker{B}}
\newcommand{\Dist}{\attacker{D}}
\newcommand{\Sim}{\attacker{S}}
\newcommand{\Ora}{\attacker{O}}
\newcommand{\Inv}{\attacker{I}}
\newcommand{\For}{\attacker{F}}

% CRYPTO SCHEMES

\newcommand{\scheme}[1]{\ensuremath{\text{#1}}}

% pseudorandom stuff
\newcommand{\prg}{\algo{PRG}}
\newcommand{\prf}{\algo{PRF}}
\newcommand{\prp}{\algo{PRP}}

% symmetric-key cryptosystem
\newcommand{\skc}{\scheme{SKC}}
\newcommand{\skcgen}{\algo{Gen}}
\newcommand{\skcenc}{\algo{Enc}}
\newcommand{\skcdec}{\algo{Dec}}

% public-key cryptosystem
\newcommand{\pkc}{\scheme{PKC}}
\newcommand{\pkcgen}{\algo{Gen}}
\newcommand{\pkcenc}{\algo{Enc}} % can also use \kemenc and \kemdec
\newcommand{\pkcdec}{\algo{Dec}}

% digital signatures
\newcommand{\sig}{\scheme{SIG}}
\newcommand{\siggen}{\algo{Gen}}
\newcommand{\sigsign}{\algo{Sign}}
\newcommand{\sigver}{\algo{Ver}}

% message authentication code
\newcommand{\mac}{\scheme{MAC}}
\newcommand{\macgen}{\algo{Gen}}
\newcommand{\mactag}{\algo{Tag}}
\newcommand{\macver}{\algo{Ver}}

% key-encapsulation mechanism
\newcommand{\kem}{\scheme{KEM}}
\newcommand{\kemgen}{\algo{Gen}}
\newcommand{\kemenc}{\algo{Encaps}}
\newcommand{\kemdec}{\algo{Decaps}}

% identity-based encryption
\newcommand{\ibe}{\scheme{IBE}}
\newcommand{\ibesetup}{\algo{Setup}}
\newcommand{\ibeext}{\algo{Ext}}
\newcommand{\ibeenc}{\algo{Enc}}
\newcommand{\ibedec}{\algo{Dec}}

% hierarchical IBE (as key encapsulation)
\newcommand{\hibe}{\scheme{HIBE}}
\newcommand{\hibesetup}{\algo{Setup}}
\newcommand{\hibeext}{\algo{Extract}}
\newcommand{\hibeenc}{\algo{Encaps}}
\newcommand{\hibedec}{\algo{Decaps}}

% binary tree encryption (as key encapsulation)
\newcommand{\bte}{\scheme{BTE}}
\newcommand{\btesetup}{\algo{Setup}}
\newcommand{\bteext}{\algo{Extract}}
\newcommand{\bteenc}{\algo{Encaps}}
\newcommand{\btedec}{\algo{Decaps}}

% trapdoor functions
\newcommand{\tdf}{\scheme{TDF}}
\newcommand{\tdfgen}{\algo{Gen}}
\newcommand{\tdfeval}{\algo{Eval}}
\newcommand{\tdfinv}{\algo{Invert}}
\newcommand{\tdfver}{\algo{Ver}}

%%% PROTOCOLS

\newcommand{\out}{\text{out}}
\newcommand{\view}{\text{view}}

%%% COMMANDS FOR LECTURES/HOMEWORKS

\newcommand{\lecheader}{%
  \chead{\large \textbf{Lecture \lecturenum\\\lecturetopic}}

  \lhead{\small
    \textbf{\href{http://www.cims.nyu.edu/~regev/teaching/crypto_fall_2016/}{Introduction to Cryptography}\\Courant, Fall 2016}}

  \rhead{\small \textbf{Instructor:
      \href{http://www.cims.nyu.edu/~regev/}{Oded Regev}\\Scribe:
      \scribename}}

  \setlength{\headheight}{20pt}
  \setlength{\headsep}{16pt}
}

\newcommand{\hwheader}{%
  \chead{\Large \textbf{Homework \hwnum}}

  \lhead{\small
    \textbf{\href{http://www.cims.nyu.edu/~regev/teaching/crypto_fall_2016/}{Introduction to Cryptography}\\Courant, Fall 2016}}

  \rhead{\small \textbf{Instructor:
      \href{http://www.cims.nyu.edu/~regev/}{Oded Regev}\\Student: \studentname}}

  \setlength{\headheight}{20pt}
  \setlength{\headsep}{16pt}
  
  \headrule


  \newcommand{\nameref}{YOUR NAME HERE}
  \ifx\nameref\studentname
    \PackageWarning{}{You forgot to fill in your name!!}
  \else    
  \fi
}

\newcommand{\examheader}{%
  \chead{\Large \textbf{Exam \\ \duedate}}
  
  \lhead{\small
    \textbf{\href{http://www.cims.nyu.edu/~regev/teaching/crypto_fall_2016/}{Introduction to Cryptography}\\Courant, Fall 2016}}

  \rhead{\small \textbf{Instructor:
      \href{http://www.cims.nyu.edu/~regev/}{Oded Regev}}}

  \setlength{\headheight}{20pt}
  \setlength{\headsep}{16pt}
  
  \headrule
}


\newcommand{\duetext}{%
\noindent Homework is due by \textbf{7am of 
  \duedate}. Send by email to both ``regev'' (under the cs.nyu.edu domain) and ``avt237@nyu.edu'' with subject line ``CSCI-GA 3210 Homework \hwnum'' and name the attachment ``\studentname~HW\hwnum.tex/pdf''. There is no need to print it.
	Start early!
}

\newcommand{\instructionstext}{%
{\small \medskip
\noindent \textbf{Instructions.} Solutions must be typeset in \LaTeX\
(a template for this homework is available on the course web page).
Your work will be graded on \emph{correctness}, \emph{clarity}, and
\emph{conciseness}.  You should only submit work that you believe to
be correct; if you cannot solve a problem completely, you will get
significantly more partial credit if you clearly identify the gap(s)
in your solution.  It is good practice to start any long solution with
an informal (but accurate) ``proof summary'' that describes the main
idea. 

\medskip \noindent
You are expected to read all the hints either before or after submission,
but before the next class.


\medskip
\noindent You may collaborate with others on this problem set and
consult external sources.  However, you must \textbf{\emph{write your
    own solutions}}. You must also \textbf{\emph{list your
    collaborators/sources}} for each problem. }
}
  
	
\newcommand{\hint}[1]{\href{http://www.cims.nyu.edu/~regev/cgi-bin/hints/index.html?#1}{{\textcolor{light-gray}{I need a hint! (ID #1)}}}}

\newcommand{\hinttext}[2]{\href{http://www.cims.nyu.edu/~regev/cgi-bin/hints/index.html?#1}{{\textcolor{light-gray}{#2 (ID #1)}}}}

\newcommand{\hintcost}[2]{\href{http://www.cims.nyu.edu/~regev/cgi-bin/hints/index.html?#1}{{\textcolor{light-gray}{I need a hint for #2 points! (ID #1)}}}}
	
\newcommand{\moreinfo}[1]{\href{http://www.cims.nyu.edu/~regev/cgi-bin/hints/index.html?#1}{{\textcolor{light-gray}{I'm done solving and want to know more! (ID #1)}}}}

% The solution macro

\newcommand{\checkfor}[1]{%
  \ifcsname#1\endcsname%
    \csname#1\endcsname %
  \fi%
}

\newenvironment{mysolution}[1]
    {\begin{solution}
    \checkfor{#1}
    }
    { 
    \end{solution}
    }

% VARIABLES

\newcommand{\hwnum}{2}
\newcommand{\duedate}{Sep 26} % changing this does not change the
                                % actual due date :)
\newcommand{\studentname}{YOUR NAME}

% END OF SUPPLIED VARIABLES

\hwheader                       % execute homework commands

\begin{document}

\pagestyle{head}                % put header on every page

\duetext

\instructionstext

% QUESTIONS START HERE.  PROVIDE SOLUTIONS WITHIN THE "solution"
% ENVIRONMENTS FOLLOWING EACH QUESTION.

\begin{questions}


%%%%%%%%%%%%%%%%%%%%%%%%%%%%%%%%%%%%%%%%%%%%%%%%%%%%%%%%%%%%%%%%%%%%%%%%%%%%%%%%%%%%%%%%
%%%%%%%%%%%%%%%%%%%%%%%%%%%%%%%%%%%%%%%%%%%%%%%%%%%%%%%%%%%%%%%%%%%%%%%%%%%%%%%%%%%%%%%%
%%%%%%%%%%%%%%%%%%%%%%%%%%%%%%%%%%%%%%%%%%%%%%%%%%%%%%%%%%%%%%%%%%%%%%%%%%%%%%%%%%%%%%%%




%%%%%%%%%%%%%%%%%%%%%%%%%%%%%%%%%%%%%%%%%%%%%%%%%%%%%%%%%%%%%%%%%%%%%%%%%%%%%%%%%%%%%%%%
%%%%%%%%%%%%%%%%%%%%%%%%%%%%%%%%%%%%%%%%%%%%%%%%%%%%%%%%%%%%%%%%%%%%%%%%%%%%%%%%%%%%%%%%
%%%%%%%%%%%%%%%%%%%%%%%%%%%%%%%%%%%%%%%%%%%%%%%%%%%%%%%%%%%%%%%%%%%%%%%%%%%%%%%%%%%%%%%%
%{
%\renewcommand*{\thefootnote}{$\clubsuit$}
  %\question\label{qu:weakvsstrongowf} 
	%\emph{(Weak vs strong one-way functions.\footnote{Again, this is a question meant to encourage you to think; you are not required to solve it fully, but you are required to demonstrate that you thought about it seriously.})} Try to complete the proof from class, showing that the existence of weak one-way functions implies the existence of strong one-way functions. At the very least, formally write what needs to be proven, and suggest ways to prove it, or explain attempts that do not work.
    %\begin{solution}
      %
    %\end{solution}
%}

{
\renewcommand*{\thefootnote}{$\clubsuit$}
  \question[3]
	\emph{(Weak vs strong one-way functions.\footnote{Again, this is a question meant to encourage you to think; you are not required to solve it fully, but you are required to demonstrate that you thought about it seriously.})} Recall that we say that $f: \bit^n \to \bit$ is a one-way function if there is an efficient algorithm for computing it, and moreover, for any PPT algorithm $I$,
	\begin{align}\label{eq:owf}
	\Pr_{x \in \bit^n} [ I(1^n, f(x)) \in f^{-1}(f(x)) ] \in \negl(n) \;,
	\end{align}
	where the $1^n$ is simply a convenient hack for allowing $I$ to run in time $\poly(n)$ (which would not be the case
	otherwise if the output of $f$ happens to be short). One can also consider a variant of this definition, known as a \emph{weak} one-way function, 
	saying that there exists a constant $c>0$ such that for any PPT $I$, Equation \eqref{eq:owf} holds with $ < 1-n^{-c}$ instead of
	$\in \negl(n)$. As their names suggest, any (strong) one-way function is also a weak one-way function (make sure you see why). 
	Can you construct an example of a weak one-way function that is not a strong one-way function? (You can assume that strong one-way functions exist)
	Can you think of a way to create a strong one-way function from a weak one-way function?
		\begin{mysolution}{weakvsstrong}
     Note that for any $c>0$, we have that $1-\frac{1}{n^{c-1}} \leq 1 - \frac{1}{n^{c}}, \forall n \geq1$. This means that given our choice of c, if we can find a function that succeeds with probability  $1-\frac{1}{n^{c-1}} $, then it is a one-way function. Choose $c=1$, so we are looking for a function whose inversion success probability is  $\frac{1}{n}$.  In this case there happens to be a function with this success probability: a function that maps all non-prime numbers to themselves, and all prime numbers to the next greatest prime. Since it there is no generally efficient way to compute primality, and even more difficult to find the next largest prime, the best guess would be to always guess the same number that you are given. This choice will never be correct if the number is prime, and always correct if it is non-prime. Since there are $O(\frac{1}{log(N)}) = O(\frac{1}{n})$ fraction of primes less than N, this is a weakly one-way function. However, it is not strongly one-way, as this success probability is clearly not negligible (the limit of success for n large isn't even 0). In order to construct a strong one-way funciton from a weak one, we might consider requiring that the adversary determine if a list of n-bit numbers are prime, with the same mapping on each component as before. If we allow the list to be of length greater than $c n log(n)$ (where $c>0$ is from the $x^{-c}$ in the definition of negligible functions). The probability of getting all of them right is then less than $(1-\frac{1}{n})^{c n log (n)} \approx n^c, \forall n>0$. Thus the probability is negligible, and the new function is strong one-way.
    \end{mysolution}
}


%%%%%%%%%%%%%%%%%%%%%%%%%%%%%%%%%%%%%%%%%%%%%%%%%%%%%%%%%%%%%%%%%%%%%%%%%%%%%%%%%%%%%%%%
%%%%%%%%%%%%%%%%%%%%%%%%%%%%%%%%%%%%%%%%%%%%%%%%%%%%%%%%%%%%%%%%%%%%%%%%%%%%%%%%%%%%%%%%
%%%%%%%%%%%%%%%%%%%%%%%%%%%%%%%%%%%%%%%%%%%%%%%%%%%%%%%%%%%%%%%%%%%%%%%%%%%%%%%%%%%%%%%%

  \question 
	\emph{(Fun with one-way functions.)}
  \begin{parts}

    \part[2] Assume we modify the definition of a one-way function by allowing the adversary to output a \emph{list} of supposed preimages, and he wins if at least one of them is a valid preimage (and as before the winning probability of any efficient adversary should be negligible). How does this modified definition compare with the original one? Formally prove your answer.
		\begin{mysolution}{funwithowfa}
    Let $P(x,n) =\Pr_{x \in \bit^n} [ I(1^n, f(x)) \in f^{-1}(f(x)) ] \in \negl(n) \;$, and let $l$ be the length of the list. Then the probability of guessing correctly at least once out of $l$ independent trials (i.e., assuming there is no improvment or deterioration on the guessing for successive guesses in the list) is $1-(1-P(x,n))^l)$. If this function is negligible,  this tells us that $ \lim_{n \to \infty} 1-(1-P(x,n))^l) n^c =0, \forall c>0$. Rearranging the equation and then taking the $l$-th root (which is preserved under taking the limit, since it is a convex function), we get  $ \lim_{n \to \infty} P(x,n) c^{\frac{c}{l}} = 0, \forall \frac{c}{l}>0$. In other words, since all of the steps above were double implications, we can see that the two definitions of one-way functions are completely equivalent.
    \end{mysolution}


    \part[2]\footnote{A question from Dodis's class\label{fn:dodis}} 
		For a security parameter $n$, define $f:\{2^{n-1},\ldots,2^n\} \to \{1,\ldots,2^{2n}\}$ by $f(x)=x^2$ (over the integers). Is it a one-way function? (Rabin's function is similar, except it's done in $\Z_N$)
		\begin{mysolution}{funwithowfb}
      No, this is not a one-way function. The inverse function, which is the square root, can be found using a binary search/bisection algorithm . We can always evaluate whether some number $n$ is greater than, less than, or equal to $\sqrt{f(x)}$, so we have all the tools we need to perform a binary search in $O(n)$ time. This is not a negligible function of n, so the function $f(x)$ is not one-way.
    \end{mysolution}


    \part[4]\footnote{A question from Peikert's class\label{fn:peikert}} Suppose that $f : \bit^{*} \to \bit^{*}$ is such that
    $\len{f(x)} \leq c \log \len{x}$ for every $x \in \bit^{*}$, where
    $c > 0$ is some fixed constant.  (Here $\len{\cdot}$ denotes the
    length of a string.)
    Prove that $f$ is \emph{not} a one-way function.  
		%(You may use a
    %\emph{non-uniform} inverter in your solution; for one bonus point,
    %use a uniform one. \hinttext{84841}{Click to read about uniform vs non-uniform})

\begin{mysolution}{funwithowfc}
 We assume that $\exists c>0 | \forall x \in \bit^{*} : |f(x)| \leq c log(|x|)$.  Let $P(x)$ be the probability of guessing an inverse that is in the preimage. Suppose we are given some known f(x) (with x unknown), and then guess an $x^{*} : c log|x^{*}| >|f(x^{*})|$, but is as small as possible, i.e. $x^{*} \leq x$. Then $|f(x^{*})| \leq |f(x)|$. Since there are no more than $2^{|f(x)|}$ possible images of $x^{*}$, the probability $P[f(x) = f(x^{*})] \geq 2^{-|f(x)|}$ Since this condition is the same as saying that your guess was in the preimage, we have that $\exists c>0 : \forall x, P(x) \geq 2^{-|f(x)|} \geq 2^{-c log |x|} = |x|^{-c}$, where we have again used the fact that $ |f(x)| \leq c log(|x|)$. Since negligible functions require that $\forall c>0, \exists x : P(x) <  |x|^{-c}$ (in fact the definition requires that this be true $\forall x$ larger than some $N_c$), our guessing probability is not negligible, and therefore the function is not one-way.
    \end{mysolution}


    \part[5]\cref{fn:dodis} 
		Assume $g:\bit^n \to \bit^n$ is a one-way function. Is the function $f:\bit^{2n} \to \bit^{2n}$ defined by
		$f(x_1,x_2) = (g(x_1),g(x_1 \oplus x_2))$ necessarily also a one-way function?
		\begin{mysolution}{funwithowfd}
      Yes. The result follows directly from the fact that $x_1$ and $x_1 \oplus x_2$ are pairwise independent We are trying to show that $P[x_1 \in g^{-1}(g(x_1)) \wedge (x_1 \oplus x_2) \in g^{-1}(g(x_1 \oplus x_2))] \in negl(x).$ Note that because  $x_1$ and $x_1 \oplus x_2$ are pairwise independent, this joint probability is just the product $P[x_1 \in g^{-1}(g(x_1)) ]* P[(x_1 \oplus x_2) \in g^{-1}(g(x_1 \oplus x_2))]$. Since the product of two negligible functions is also negligible, the function $f(x_1,x_2)$ is one-way.
    \end{mysolution}


    \part[3] (bonus\footnote{By Bao Feng, as appears in Goldreich's book}) Show that there exists a one-way function
		$f:\bit^n \to \bit^n$ for which the function $f'(x):=f(x)\oplus x$ is \emph{not} one-way. You can assume the 
		existence of a one-way function $g:\bit^n \to \bit^n$ for all $n$.
		\hintcost{82778}{1/2}

		\begin{mysolution}{funwithowfe}
      Yes. Choose some a secret key $k \in \bit^n$, and define the function $f:\bit^2n \to \bit^2n, f(x,k) = (x \oplus k,0)$. Since XORs are all pairwise independent, having $x \oplus k$ does not tell us anything about either k or x, and thus the function is one-way. However, $x \oplus f(x,k) = ( k,x) $. This tells us the full information about both x and k, and therefore allows us to choose an element in the preimage every time. So the function is not one-way.
      \end{mysolution}



  \end{parts}


%%%%%%%%%%%%%%%%%%%%%%%%%%%%%%%%%%%%%%%%%%%%%%%%%%%%%%%%%%%%%%%%%%%%%%%%%%%%%%%%%%%%%%%%
%%%%%%%%%%%%%%%%%%%%%%%%%%%%%%%%%%%%%%%%%%%%%%%%%%%%%%%%%%%%%%%%%%%%%%%%%%%%%%%%%%%%%%%%
%%%%%%%%%%%%%%%%%%%%%%%%%%%%%%%%%%%%%%%%%%%%%%%%%%%%%%%%%%%%%%%%%%%%%%%%%%%%%%%%%%%%%%%%



  \question[6] 
	\emph{(Worst-case to average-case reduction.\cref{fn:peikert})}
    Let $N$ be the product of two distinct $n$-bit primes,
    and suppose there is an efficient algorithm $\Adv$ that computes
    square roots on a noticeable fraction of quadratic residues mod
    $N$: \[ \Pr_{y \gets \QRN^{*}}[\Adv(N, y) \in \sqrt{y} \bmod N] =
    \delta \geq 1/\poly(n). \] Construct an efficient algorithm
    $\AdvB$ that, using $\AdvA$ as an oracle, computes the square root
    of \emph{any} $y \in \QRN^{*}$ with \emph{overwhelming} probability
    (solely over the random coins of $\Adv$ and $\AdvB$).  That is,
    for every $y \in \QRN^{*}$, it should be the case that
    \[ \Pr[\AdvB^{\AdvA}(N, y) \in \sqrt{y} \bmod N] = 1 - \negl(n). \]
    Explain in your own words why such reductions are known as worst-case to average-case reductions.
		\begin{mysolution}{worstaveragecasereduction}
      Now suppose we have some $c>0$ as in the definition of negligible functions, and we wish to find an algorithm such that that the failure probability is negligible, i.e. that $\lim_{n \to \infty} \Pr[\AdvB^{\AdvA}(N, y) \in \sqrt{y} \bmod N] x^c = 0, \forall c>0$. In this case, consider the following strategy: Repeat the algorithm $\AdvA$ a total of $(c+1)*poly(n) * log(n)$ times. After each iteration we can efficiently compute whether our square root was actually valid. So the algorithm is successful if any of the computed roots are correct. The probability that all of the guesses are wrong is $(1-\frac{1}{poly(n)})^{(c+1) * poly(n) * log(n))} = e^{- (c+1) log(n)} = n^{-(c+1)}$ As we can see, $\lim_{n \to \infty} n^{-(c+1)} n^{c} = 0$, and therefore the failure probability is negligible. Note that this algorithm is still efficien, because it is repeated a polynomial number of times. The reason why these reduction might be called worst case to average case reductions is that the algorithm went from being correct only on very rare occasions (worst case) to being correct nearly all of the time.
    \end{mysolution}



%%%%%%%%%%%%%%%%%%%%%%%%%%%%%%%%%%%%%%%%%%%%%%%%%%%%%%%%%%%%%%%%%%%%%%%%%%%%%%%%%%%%%%%%
%%%%%%%%%%%%%%%%%%%%%%%%%%%%%%%%%%%%%%%%%%%%%%%%%%%%%%%%%%%%%%%%%%%%%%%%%%%%%%%%%%%%%%%%
%%%%%%%%%%%%%%%%%%%%%%%%%%%%%%%%%%%%%%%%%%%%%%%%%%%%%%%%%%%%%%%%%%%%%%%%%%%%%%%%%%%%%%%%

{
\renewcommand*{\thefootnote}{$\clubsuit$}
  \question\label{qu:prg} 
	\emph{(PRG)} Try to think how to precisely define the property that a function $f:\{0,1\}^n \to \{0,1\}^{n+1}$ satisfies that $f(U)$ ``looks" like a uniform string in $\{0,1\}^{n+1}$ where $U$ is sampled uniformly from $\{0,1\}^n$. There is no need to write down your solution: just think about it in preparation for Monday's class. Such efficiently computable functions are known as \emph{pseudorandom generators}. 
}




\end{questions}

\end{document}

%%% Local Variables: 
%%% mode: latex
%%% TeX-master: t
%%% End: 
